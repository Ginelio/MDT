\gr
\chapter{Υπό Μελέτη Μικροδίκτυο}
Το μικροδίκτυο που εξετάζεται στην παρούσα μελέτη αποτελείται από ανανεώσιμες πηγές ενέργειας και τεχνολογίες αποθήκευσης. Συγκεκριμένα, η παραγωγή ενέργειας προέρχεται από μία συστοιχία φωτοβολταϊκών πάνελ, μία ανεμογεννήτρια, ενώ η αποθήκευση ενέργειας υλοποιείται μέσω συστοιχιών μπαταριών και ενός συστήματος υδρογόνου. Το σύστημα υδρογόνου περιλαμβάνει τρεις κύριες μονάδες: τη μονάδα ηλεκτρόλυσης για την παραγωγή υδρογόνου, τη μονάδα αποθήκευσης και τη μονάδα κυψελών καυσίμου για την παραγωγή ηλεκτρικής ενέργειας. Αυτή η δομή επιτρέπει την ολοκληρωμένη διαχείριση των ανανεώσιμων πηγών, συμβάλλοντας στην ενεργειακή αυτονομία του μικροδικτύου.

Στο πλαίσιο της παρούσας διπλωματικής εργασίας, η λειτουργία του μικροδικτύου προκύπτει από την επίλυση ενός γραμμικού προβλήματος βελτιστοποίησης, όπου λαμβάνονται υπόψη οι περιορισμοί που σχετίζονται με τα χαρακτηριστικά των μονάδων αποθήκευσης και τις απαιτήσεις του συστήματος. Η βελτιστοποίηση επικεντρώνεται στα κόστη λειτουργίας των μονάδων αποθήκευσης, ενώ η παραγωγή ενέργειας από τις ανανεώσιμες πηγές θεωρείται δεδομένη, καθώς οι μονάδες παραγωγής λειτουργούν πάντοτε στο μέγιστο επίπεδο τους, ανάλογα με τις καιρικές συνθήκες. Ο βασικός στόχος είναι να καθοριστεί ποιες μονάδες αποθήκευσης είναι οικονομικά αποδοτικό να ενεργοποιηθούν, ώστε να επιτευχθεί η ελαχιστοποίηση του συνολικού κόστους λειτουργίας του μικροδικτύου.

% Ένα χαρακτηριστικό παράδειγμα λειτουργίας ενός τέτοιου μικροδικτύου είναι το μικροδίκτυο της Ξάνθης, που αναπτύχθηκε στο πλαίσιο του έργου \en X-FLEX \cite{10137658}. \gr Η πλεονάζουσα ενέργεια που παράγεται από τα φωτοβολταϊκά και τις ανεμογεννήτριες χρησιμοποιείται για τη φόρτιση των μπαταριών του συστήματος. Εφόσον οι μπαταρίες είναι πλήρως φορτισμένες, η επιπλέον ενέργεια τροφοδοτεί τη μονάδα ηλεκτρόλυσης για την παραγωγή υδρογόνου, το οποίο αποθηκεύεται σε συμπιεσμένη μορφή σε ειδικές δεξαμενές για μελλοντική χρήση. Σε περίπτωση που και οι μπαταρίες και οι δεξαμενές υδρογόνου είναι πλήρεις, η πλεονάζουσα ενέργεια πωλείται στο δίκτυο.

% Όταν η ζήτηση ενέργειας του μικροδικτύου δεν μπορεί να καλυφθεί από τις ανανεώσιμες πηγές, ενεργοποιείται η αποθηκευμένη ενέργεια. Αρχικά χρησιμοποιούνται οι μπαταρίες, ενώ εάν αυτές εξαντληθούν, το υδρογόνο από τις δεξαμενές μετατρέπεται σε ηλεκτρική ενέργεια μέσω της μονάδας κυψελών καυσίμου. Εάν η αποθηκευμένη ενέργεια δεν επαρκεί για την κάλυψη της ζήτησης, το μικροδίκτυο λαμβάνει ενέργεια από το κεντρικό δίκτυο. Αυτή η διαδικασία δημιουργεί νέες προοπτικές στην αγορά ενέργειας, συμβάλλοντας παράλληλα στη μείωση των εκπομπών αερίων του θερμοκηπίου και στην καταπολέμηση της κλιματικής αλλαγής.

\section{Φωτοβολταϊκό σύστημα}
Η ισχύς εξόδου ενός φωτοβολταϊκού συστήματος υπολογίζεται βάσει της σχέσης \cite{CAU2014820}:
\begin{equation}
    P_{PV}=G\cdot A_{PV} \cdot N_{PV} \cdot η_{PV} \label{P_PV}
\end{equation}
όπου:
\begin{itemize}
  \item[] $P_{PV}$, η ισχύς εξόδου ενός φωτοβολταϊκού συστήματος σε $kW$ 
  \\
  \item[] $G$, η προσπίπτουσα ηλιακή ακτινοβολία σε $W/m^2$ 
  \\
  \item[] $A_{PV}$, η επιφάνεια ενός φωτοβολταϊκού πάνελ σε $m^2$
  \\
  \item[] $N_{PV}$, ο συνολικός αριθμός των πάνελ του φωτοβολταϊκού συστήματος 
  \\
  \item[] $η_{PV}$, η απόδοση ενός ηλιακού πάνελ 
\end{itemize}

Η απόδοση κάθε ηλιακού πάνελ υπολογίζεται ως εξής \cite{CAU2014820}:
\begin{equation}
    \eta_{PV} = \eta_{PV,REF} \cdot \left[ 1 - \alpha \cdot \left( T_{AMB} + G \cdot \frac{NOCT - 20}{800} - T_{REF} \right) \right] \label{\eta_PV}
\end{equation}
όπου:
\begin{itemize}
  \item[] $\eta_{PV}$, η απόδοση ενός ηλιακού πάνελ σε συγκεκριμένες συνθήκες
  \\
  \item[] $\eta_{PV,REF}$, η απόδοση ενός ηλιακού πάνελ όταν $G = 1000 W/m^2$ και $T_{REF}=25^oC$)
  \\
  \item[] $T_{AMB}$, η θερμοκρασία περιβάλλοντος σε $^oC$
  \\
  \item[] α, ο συντελεστής θερμοκρασίας σε $1/K$ 
  \\
  \item[] $NOCT$, η ονομαστική θερμοκρασία λειτουργίας της μονάδας σε $^oC$ 
  \\
  \item[] $T_{REF}$, η θερμοκρασία αναφοράς της μονάδας, $Τ_{REF}=25^oC$ 
\end{itemize}
Τα παραπάνω στοιχεία δίνονται από τον κατασκευαστή της φωτοβολταϊκής μονάδας \cite{sunpower} και καταγράφονται στον Πίνακα \ref{tab:pv_data} \cite{CAU2014820}. 

Για την επίλυση του προβλήματος βελτιστοποίησης θεωρούμε ότι το φωτοβολταϊκό σύστημα λειτουργεί διαρκώς στο \en MPP (Maximum Power Point). \gr Οι παράμετροι που καθορίζουν την απόδοση του φωτοβολταϊκού συστήματος, $η_{PV}$, είναι η ηλιακή ακτινοβολία και η θερμοκρασία περιβάλλοντος. Τα μετεωρολογικά αυτά δεδομένα εξήχθησαν για την περιοχή της νότιας Κρήτης, την ημέρα 23/07/2023 \cite{solcasttoolkit}. 

\section{Ανεμογεννήτρια}
Η ισχύς εξόδου μίας ανεμογεννήτριας υπολογίζεται βάσει της εξίσωσης \cite{mathematical}:
\begin{equation}
    P_{WT} = 
        \begin{cases} 
        0 & 
        \text{αν } v(t) < v_{cut-in} \text{ ή } v(t) \geq v_{\text{\en cut-out}} \\

        \frac{{(v(t) - v_{\text{\en cut-in}})}}{{(v_{\text{\en rated}} - v_{\text{\en cut-in}})}} \cdot P_{rated} & 
        \text{αν } v_{cut-in} \leq v(t) < v_{rated} \\

        P_{rated} & 
        \text{αν } \en v_{rated} \leq v(t) \leq v_{cut-out} \\

        0 & 
        \text{σε κάθε άλλη περίπτωση} 
        \end{cases} 
        \label{P_WT}
\end{equation}

όπου:
\begin{itemize}
    \item[] $P_{WT}$, η ισχύς της ανεμογεννήτριας σε $kW$
    \\
    \item[] $v$, η ταχύτητα του ανέμου σε $m/s$
    \\
    \item[] $v_{cut-in}$, η ταχύτητα εισόδου σε $m/s$
    \\
    \item[] $v_{rated}$, η ονομαστική ταχύτητα σε $m/s$
    \\
    \item[] $v_{cut-out}$, η ταχύτητα εξόδου σε $m/s$
    \\
    \item[] $P_{rated}$, η ονομαστική ισχύς σε $kW$
\end{itemize}

Η ταχύτητα εισόδου, η ταχύτητα εξόδου καθώς και η ονομαστική ισχύς της ανεμογεννήτριας δίνονται από τον κατασκευαστή της ανεμογεννήτριας \cite{CAU2014820}. Η ταχύτητα ανέμου δίνεται από τα μετεωρολογικά που εξήχθησαν για την περιοχή της νότιας Κρήτης, την ημέρα 23/07/2023 \cite{solcasttoolkit}. Συγκεκριμένα ανά ώρα έχουμε μεταβολή στην τιμή της ταχύτητας του ανέμου. Τα δεδομένα αυτά παρουσιάζονται παρακάτω στον Πίνακα \ref{tab:data}.

Λαμβάνοντας λοιπόν τα δεδομένα της ταχύτητας του ανέμου, μπορούμε να κατασκευάσουμε την καμπύλη ισχύος της ανεμογεννήτριας και να λάβουμε την ισχύ εξόδου της μονάδας σε κάθε χρονική στιγμή. Η ανεμογεννήτρια του μικροδικτύου αποτελείται από τρία πτερύγια, είναι κάθετου άξονα και τα χαρακτηριστικά της δίνονται στον Πίνακα \ref{tab:wt_data} \cite{CAU2014820}.

\section{Μπαταρίες}
Η χρήση μπαταριών για την κάλυψη της ζήτησης όταν δεν υπάρχει παραγωγή είναι μία από τις πιο συνηθισμένες λύσεις σε μικροδίκτυα με ανανεώσιμους διεσπαρμένους πόρους. Ο συνδυασμός τους με την αποθήκευση υδρογόνου αποτελεί έναν διαφορετικό τρόπο λειτουργίας που συμβάλλει στην μεγιστοποίηση του χρόνου ζωής των μπαταριών, αλλά και την αποδοτικότητά τους. 

Η ενέργεια που μπορεί να προσφέρει μία μπαταρία εξαρτάται από την κατάσταση φόρτισής της, το λεγόμενο \en "state of charge" (SOC). \gr Αυτό ισούται με τον λόγο της αποθηκευμένης ενέργειας συναρτήσει της ονομαστικής χωρητικότητας της μπαταρίας και υπολογίζεται παρακολουθώντας την ισχύ φόρτισης και εκφόρτισης της μπαταρίας μέσα στον χρόνο \cite{CAU2014820}:

\begin{equation}
    SOC(t)=SOC(t-1)+\frac{(P_{B,CH}(t) \cdot η_{B,CH}-P_{B,DIS}(t) \cdot η_{B,DIS}) \cdot Δt}{Ν_{Β}U_{Β}Q_{Β}} \label{SOC}
\end{equation}
όπου:
\begin{itemize}
  \item[] $P_{B,CH}(t)$, η ισχύς φόρτισης την ώρα t σε $W$
  \\
  \item[] $P_{B,DIS}(t)$, η ισχύς εκφόρτισης την ώρα t σε $W$
  \\
  \item[] $η_{B,CH}$, η απόδοση της μπαταρίας κατά την φόρτιση 
  \\
  \item[] $η_{B,DIS}$, η απόδοση της μπαταρίας κατά την εκφόρτιση 
  \\
  \item[] $Δt$, το χρονικό βήμα μελέτης στάθμης φόρτισης, το οποίο ισούται με 1 $h$
  \\
  \item[] $N_B$, ο αριθμός των μπαταριών 
  \\
  \item[] $Q_B$, η ονομαστική χωρητικότητα της μπαταρίας σε $Ah$ 
  \\
  \item[] $U_B$, η ονομαστική τάση της μπαταρίας σε $V$ 
  \\
\end{itemize}

Το σύστημα μπαταριών αποτελείται από συστοιχίες μπαταριών μολύβδου οξέος και τα χαρακτηριστικά του καταγράφονται στον Πίνακα \ref{tab:bat_data} \cite{CAU2014820}.

Το κόστος λειτουργίας και συντήρησης της συστοιχίας μπαταριών δεν υπολογίζεται στην παρούσα διπλωματική εργασία. Θεωρείται ότι μεγαλύτερη βαρύτητα έχει ο εκτιμώμενος χρόνος ζωής του συστήματος, ο οποίος υπολογίζεται τόσο για την φόρτιση όσο και για την εκφόρτιση σύμφωνα με τις παρακάτω εξισώσεις αντίστοιχα:  
    \begin{equation}
        L_{B,CH} = \frac{N_B U_B Q_B}{P_{B,CH}(t)}N_{CYCLES} \label{L_B_CH}
    \end{equation}
    \begin{equation}
        L_{B,DIS} = \frac{N_B U_B Q_B}{P_{B,DIS}(t)/η_{B,DIS}}N_{CYCLES} \label{L_B_DIS}
    \end{equation}
όπου:
\begin{itemize}
    \item[] $L_{B,CH}$, ο χρόνος ζωής της μπαταρίας κατά την φόρτιση σε $h$
    \\
    \item[] $L_{B,DIS}$, ο χρόνος ζωής της μπαταρίας κατά την εκφόρτιση σε $h$
    \\
    \item[] $N_{CYCLES}$, ισοδύναμοι κύκλοι φόρτισης και εκφόρτισης της μπαταρίας 
\end{itemize}

\section{Μονάδα υδρογόνου}
Η μονάδα υδρογόνου αποτελείται από τρεις ξεχωριστές μονάδες που συνδέονται μεταξύ τους:
\begin{itemize}
    \item Mονάδα ηλεκτρόλυσης 
    \\
    \item Mονάδα αποθήκευσης υδρογόνου
    \\
    \item Mονάδα κυψελών καυσίμου υδρογόνου 
\end{itemize}
Η κάθε μονάδα χαρακτηρίζεται από διαφορετικές εξισώσεις, που στηρίζονται στις βασικές αρχές λειτουργίας τους. 

\subsection{Μονάδα ηλεκτρόλυσης}

Η μονάδα ηλεκτρόλυσης λειτουργεί αντίθετα από μία μονάδα κυψελών καυσίμου υδρογόνου. Έχοντας ως είσοδο το νερό και χρησιμοποιώντας την παραγόμενη ενέργεια από το μικροδίκτυο, παράγει υδρογόνο, το οποίο στη συνέχεια συμπιέζεται σε ειδικές δεξαμενές για αποθήκευση. 

Βασιζόμενοι στον νόμο του \en Faraday, \gr μπορούμε να εξάγουμε την παραγόμενη ροή υδρογόνου συναρτήσει της ηλεκτρικής ενέργειας που δίνεται από το δίκτυο στη μονάδα ηλεκτρόλυσης \cite{CAU2014820}. 
\begin{equation}
    n_{H_2,EL}=\frac{η_{EL}P_{EL}(t)\cdot3600}{LVH_{H_2}} \label{n_H2_EL}
\end{equation}
όπου:
\begin{itemize}
  \item[] $n_{H_2,EL}$, η παραγόμενη ροή υδρογόνου της μονάδας σε $mol/h$
  \\
  \item[] $η_{EL}$, η απόδοση της μονάδας, η οποία λαμβάνει υπόψιν όλες τις απώλειες (ηλεκτροχημικές, θερμοδυναμικές και επικουρικές)
  \\
  \item[] $P_{EL}(t)$, η ισχύς που παρέχεται στην μονάδα από τις ανανεώσιμες πηγές ενέργειας του μικροδικτύου την ώρα t σε $kW$
  \\
  \item[] $LVH_{H_2}$, η ελάχιστη θερμοκρασία καύσης του υδρογόνου η οποία ισούται με $240 kJ/mol$
\end{itemize}

Από τον κατασκευαστή γνωρίζουμε ότι ο ρυθμός παραγωγής του υδρογόνου στη μονάδα ηλεκτρόλυσης ισούται με $1.05 Nm^3$. Δηλαδή $1.05 m^3$ σε συνθήκες πίεσης $1 bar$ και θερμοκρασίας $0^oC$. Συνεπώς, λαμβάνοντας υπόψιν την καταστατική εξίσωση τέλειων αερίων, η μονάδα ηλεκτρόλυσης παράγει ανά ώρα:
\begin{equation}
   n_{Η_2,EL,REF}=\frac{1 bar \cdot 1.05 m^3}{0.08314 m^3bar/molK \cdot 273.15 K} \label{n_H2_EL_REF}
\end{equation}
\begin{equation}
        n_{Η_2,EL,REF}=0.46 mol
\end{equation}

Τα χαρακτηριστικά της μονάδας ηλεκτρόλυσης καταγράφονται στον Πίνακα \ref{tab:hyd_data} \cite{CAU2014820}.

\subsection{Μονάδα αποθήκευσης υδρογόνου}

Λόγω του μεγάλου όγκου του αέριου υδρογόνου, είναι σημαντικό να αποθηκεύεται σε όσο το δυνατόν υψηλότερες πιέσεις ώστε να μειώνεται ο όγκος αποθήκευσης και να αυξάνεται τελικά το ποσό καυσίμου που μπορούμε να έχουμε ως εφεδρεία.

Σύμφωνα με την καταστατική εξίσωση των αερίων υπολογίζουμε την πίεση σε κάθε χρονική στιγμή μέσα στη δεξαμενή \cite{CAU2014820}. 
\begin{equation}
    p_{H_2}(t)=p_{H_2}(t-1)+\frac{RT_{H_2}}{V_{H_2}}(n_{H_2,EL}-n_{H_2,FC}) \label{p_H2}
\end{equation}
όπου:
\begin{itemize}
  \item[] $p_{H_2}(t)$, η πίεση του υδρογόνου στη δεξαμενή την ώρα $t$ σε $bar$
  \\
  \item[] $p_{H_2}(t-1)$, η πίεση του υδρογόνου στη δεξαμενή την ώρα $(t-1)$ σε $bar$
  \\
  \item[] $R$, η παγκόσμια σταθερά των αερίων που ισούται με $0,08314\ m^3bar/molK$
  \\
  \item[] $T_{H_2}$, η μέση θερμοκρασία μέσα στη δεξαμενή σε $K$ 
  \\
  \item[] $V_{H_2}$, ο συνολικός όγκος της δεξαμενής σε $m^3$ 
\end{itemize}

Η μέγιστη πίεση του υδρογόνου είναι αρκετά χαμηλή, της τάξης των $13.8 bar$, για τον λόγο αυτό χρησιμοποιήθηκε και η καταστατική εξίσωση τελείων αερίων. Τα χαρακτηριστικά της μονάδας αποθήκευσης υδρογόνου καταγράφονται στον Πίνακα \ref{tab:hyd_data}.
        
\subsection{Μονάδα κυψελών καυσίμου υδρογόνου}

Το παραγόμενο υδρογόνο από την μονάδα ηλεκτρόλυσης τροφοδοτείται στην μονάδα κυψελών καυσίμου υδρογόνου για την παραγωγή ηλεκτρικής ενέργειας σε περιόδους αιχμής.

Το υδρογόνο που καταναλώνεται στη μονάδα κυψελών καυσίμου υδρογόνου είναι άμεσα συνδεδεμένο με την ισχύ εξόδου της μονάδας βάσει της εξίσωσης \cite{CAU2014820}:
\begin{equation}
   n_{Η_2,FC}=\frac{P_{FC}(t)\cdot3600}{\eta_{FC}LHV_{H_2}} \label{n_H2_FC}
\end{equation}
όπου:
\begin{itemize}
  \item[] $n_{Η_2,FC}$, η κατανάλωση υδρογόνου στην μονάδα σε $mol/h$
  \\
  \item[] $P_{FC}(t)$, η ισχύς εξόδου της μονάδας κυψέλης καυσίμου υδρογόνου σε $kW$
  \\
  \item[] $η_{FC}$, η απόδοση της μονάδας, η οποία λαμβάνει υπόψιν όλες τις απώλειες (ηλεκτροχημικές, θερμοδυναμικές και επικουρικές)
  \\
  \item[] $LVH_{H_2}$, η ελάχιστη θερμοκρασία καύσης του υδρογόνου η οποία ισούται με $240 MJ/kmol$ 
\end{itemize}   

Η μονάδα κυψελών καυσίμου υδρογόνου έχει τα χαρακτηριστικά που καταγράφονται στον Πίνακα \ref{tab:hyd_data}.

Από τον κατασκευαστή γνωρίζουμε ότι ο ρυθμός κατανάλωσης του υδρογόνου στη μονάδα κυψελών καυσίμου υδρογόνου ισούται με $3.90 Nm^3$. Δηλαδή $3.90 m^3$ σε συνθήκες πίεσης $1 bar$ και θερμοκρασίας $0^oC$. Συνεπώς, λαμβάνοντας υπόψιν την καταστατική εξίσωση τέλειων αερίων, η μονάδα κυψελών καυσίμου υδρογόνου καταναλώνει ανά ώρα:
\begin{equation}
    n_{Η_2,FC,REF}=\frac{1 bar \cdot 3.90 m^3}{0.08314 m^3bar/molK \cdot 273.15 K} \label{n_FC_REF}
\end{equation}
\begin{equation}
    n_{Η_2,FC,REF}=0,17 mol
\end{equation}


\section{Πλεονάζουσα \& μη αποδοθείσα ενέργεια}
Η ορθή λειτουργία ενός μικροδικτύου προϋποθέτει τη διαχείριση των ενεργειακών ροών ώστε να διατηρείται το ενεργειακό ισοζύγιο. Σε ορισμένες περιπτώσεις, καθίσταται αναγκαία η απόρριψη φορτίου ή ισχύος, προκειμένου να εξασφαλιστεί αυτή η ισορροπία. Για παράδειγμα, όταν η ζήτηση υπερβαίνει την διαθέσιμη ισχύ που παράγεται από τις ανανεώσιμες πηγές ενέργειας (ΑΠΕ) του συστήματος και οι μονάδες αποθήκευσης δεν έχουν τη δυνατότητα να καλύψουν τη διαφορά, τότε μέρος της ζήτησης δεν ικανοποιείται. Η ποσότητα της ενέργειας που δεν μπορεί να αποδοθεί ορίζεται ως \en undelivered power \gr και εκφράζεται μέσω της μεταβλητής $P_{UN}$.

Αντιθέτως, όταν η ζήτηση καλύπτεται εξ ολοκλήρου από τις ανανεώσιμες πηγές ενέργειας και οι μονάδες αποθήκευσης έχουν ήδη φτάσει στο μέγιστο της χωρητικότητάς τους, η παραγόμενη πλεονάζουσα ενέργεια δεν μπορεί να αποθηκευτεί, οδηγώντας σε μείωση της παραγωγής. Αυτή η ποσότητα ενέργειας που περισσεύει ορίζεται ως \en excess power \gr και εκφράζεται με τη μεταβλητή $P_{EX}$. Αυτό το φαινόμενο είναι σύνηθες σε περιπτώσεις όπου η παραγωγή των ΑΠΕ υπερβαίνει τη ζήτηση και το διαθέσιμο δυναμικό αποθήκευσης δεν επαρκεί για την απορρόφησή της.

Η διαχείριση τόσο της μη αποδοθείσας όσο και της πλεονάζουσας ενέργειας αποτελεί κρίσιμο παράγοντα για την επίτευξη βέλτιστης απόδοσης και αξιοπιστίας σε συστήματα μικροδικτύων, ειδικά σε εκείνα που βασίζονται σε ΑΠΕ και μονάδες αποθήκευσης ενέργειας.






