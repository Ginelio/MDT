\chapter{Βελτιστοποίηση λειτουργίας μικροδικτύου}
Το πρόβλημα μελετάται σε χρονικό ορίζοντα ενός εικοσιτετραώρου. Υπολείπονται τα δεδομένα ζήτησης της ενέργειας για κάθε ώρα.

Η παραγωγή ενέργειας στο μικροδίκτυο έρχεται από τα δύο συστήματα ανανεώσιμων πηγών ενέργειας, το φωτοβολταϊκό σύστημα και την ανεμονεννήτρια. Συνεπώς στο πρόβλημα που μελετάται δεν συμπεριλαμβάνεται κόστος καυσίμου. Η ισχύς εξόδου των μονάδων είναι συνάρτηση μόνο της ηλιακής ακτινοβολίας και της έντασης του ανέμου.

Στόχος είναι να βρεθεί το σημείο βέλτιστης λειτουργίας των μονάδων αποθήκευσης του συστήματος, τα οποία είναι οι μπαταρίες, η μονάδα ηλεκτρόλυσης και η μονάδα κυψέλης καυσίμου υδρογόνου, ώστε να ελαχιστοποιηθεί το συνολικό κόστος λειτουργίας του μικροδικτύου.

%Η ισχύς εξόδου κάθε μονάδας του μικροδικτύου μετατρέπεται στο πρόβλημα βελτιστοποίησης σε μία δυαδική μεταβλητή, που καθορίζει το status της μονάδας.

\section{Κόστoς Λειτουργίας Μικροδικτύου}
Όταν παράγεται περίσσεια ενέργειας από τις ανανεώσιμες πηγές του μικροδικτύου, ελέγχεται η αποδοτικότερη επιλογή αποθήκευσής αυτής. 

Στο παρόν μικροδίκτυο μελετάται μόνο το κόστος της αποθήκευσης της παραγόμενης ενέργειας κι όχι αυτό των μονάδων παραγωγής. Αυτό διότι οι μονάδες αυτές έχουν ένα σχεδόν σταθερό κόστος που θεωρούμε αμελητέο.

\subsection{Κόστος λειτουργίας μονάδας μπαταριών}
Το κόστος λειτουργίας των μπαταριών κατά την φόρτιση υπολογίζεται ως εξής \cite{CAU2014820}:
\begin{equation}
    C_{B,CH} = \frac{C_{B,IN}/L_{B,CH} + C_{O\&M,B}}{η_{B,CH}η_{B,DIS}}
\end{equation}
όπου:
\begin{itemize}
  \item[] $C_{B,CH}$, το κόστος λειτουργίας κατά την φόρτιση σε €
  \\
  \item[] $C_{B,IN}$, το κόστος κεφαλαίου των μπαταριών σε €
  \\
  \item[] $L_{B,CH}$, ο χρόνος ζωής της μπαταρίας κατά την φόρτιση σε $h$
  \\
  \item[] $C_{O\&M,i}$, το κόστος λειτουργίας και συντήρησης κάθε μονάδας σε $$€/h$, το οποίο για τις μπαταρίες θεωρείται αμελητέο
\end{itemize}

Ο χρόνος ζωής της μπαταρίας κατά την φόρτιση δεν υπολογίζεται σε ώρες, αλλά σε πλήρεις κύκλους φόρτισης κι εκφόρτισης \cite{CAU2014820} σύμφωνα με την εξίσωση \ref{L_B_CH}. 

Το κόστος λειτουργίας των μπαταριών κατά την εκφόρτιση υπολογίζεται ως εξής \cite{CAU2014820}:
\begin{equation}
    C_{B,DIS} = C_{B,IN}/L_{B,DIS} + C_{O\&M,B} \label{C_B_DIS}
\end{equation}
όπου:
\begin{itemize}
  \item[] $C_{B,DIS}$, το κόστος λειτουργίας κατά την εκφόρτιση σε €
  \\
  \item[] $C_{B,IN}$, το κόστος κεφαλαίου σε €
  \\
  \item[] $L_{B,DIS}$, ο χρόνος ζωής της μπαταρίας κατά την εκφόρτιση σε $h$
  \\
  \item[] $C_{O\&M,B}$, το κόστος λειτουργίας και συντήρησης της μονάδας σε €/$h$
\end{itemize}

Ο χρόνος ζωής της μπαταρίας κατά την εκφόρτιση υπολογίζεται βάσει της εξίσωσης \ref{L_B_DIS} \cite{CAU2014820}.

\subsection{Κόστος λειτουργίας μονάδας υδρογόνου}
Το κόστος λειτουργίας της μονάδας αποθήκευσης υδρογόνου υπολογίζεται με παρόμοιο τρόπο όπως στις μπαταρίες κι αφορά την παραγωγή του μέσω της μονάδας ηλεκτρόλυσης και της χρήσης του ως καύσιμο στην μονάδα κυψέλης καυσίμου υδρογόνου, \cite{CAU2014820}:

\begin{equation}
    C_{H_2,CH}=\frac{(C_{EL,IN}/L_{EL}+C_{O\&M,EL})+(C_{FC,IN}/L_{FC}+C_{O\&M,FC})\cdot Δt_{FC}/Δt}{η_{EL}\cdot η_{FC}} \label{C_H2_CH}
\end{equation}

όπου:
\begin{itemize}
  \item[-] $C_{H_2,CH}$, το κόστος λειτουργίας της μονάδας αποθήκευσης υδρογόνου σε €
  \item[-] $C_{EL,IN}$, το κόστος επένδυσης της μονάδας ηλεκτρόλυσης σε €
  \item[-] $C_{FC,IN}$, το κόστος επένδυσης της μονάδας κυψελών καυσίμου υδρογόνου σε €
  \item[-] $L_{EL}$, ο χρόνος ζωής της μονάδας ηλεκτρόλυσης σε h
  \item[-] $L_{FC}$, ο χρόνος ζωής της μονάδας κυψελών καυσίμου υδρογόνου σε h
  \item[-] $C_{O\&M,EL}$, το κόστος λειτουργίας και συντήρησης της μονάδας ηλεκτρόλυσης σε €/h
  \item[-] $C_{O\&M,FC}$, το κόστος λειτουργίας και συντήρησης της μονάδας κυψελών καυσίμου υδρογόνου σε €/h
  \item[-] $Δt_{FC}/Δt$, ώρες λειτουργίας της μονάδας κυψελών καυσίμου υδρογόνου 
\end{itemize}

Οι ώρες λειτουργίας της μονάδας κυψελών καυσίμου υδρογόνου υπολογίζονται υποθέτοντας ότι η μονάδα λειτουργεί στην ονομαστική της ισχύ και καταναλώνει όλο το υδρογόνο που παράγεται στη μονάδα ηλεκτρόλυσης κατά το χρονικό βήμα Δt \cite{CAU2014820}.
\begin{equation}
    Δt_{FC}/Δt = η_{EL} \cdot η_{FC} \cdot \frac{P_{H_2,CH}(t)}{P_{NOM,FC}}
\end{equation}
όπου:
\begin{itemize}
 \item[-] $Δt$, το χρονικό βήμα μελέτης στάθμης φόρτισης, το οποίο ισούται με 1 $h$
\end{itemize}

Εάν η ζήτηση ενέργειας πρέπει να εξυπηρετηθεί από την μονάδα αποθήκευσης υδρογόνου, το κόστος λειτουργίας υπολογίζεται ως εξής \cite{CAU2014820}:
\begin{equation}
    C_{FC} = C_{FC,IN}/L_{FC} + C_{O\&M,FC}
\end{equation}

Για την αποφυγή απόρριψης ή περικοπής φορτίου, δημιουργούνται εικονικά κόστη που σχετίζονται με την απορριπτόμενη ενέργεια, $C_{UN}$, και την περίσσεια ενέργειας, $C_{EX}$. Συγκεκριμένα, τα κόστη αυτά έχουν οριστεί 0.005 €/W h.

\subsection{Συνολικό κόστος λειτουργίας μικροδικτύου}
Το συνολικό κόστος λειτουργίας του μικροδικτύου διαμορφώνεται ως εξής \cite{CAU2014820}:
\begin{equation}
    C=\frac{1}{η} \sum_{i} (\frac{C_{IN,i}}{L_i} + C_{O\&M,i}) 
\end{equation}
όπου:
\begin{itemize}
  \item[-] $C$, το κόστος λειτουργίας του μικροδικτύου σε €
  \item[-] $η$, δείκτης απόδοσης μίας μονάδας αποθήκευσης (roundtrip efficiency), ο οποίος μας δίνει τις απώλειες του συστήματος
  \item[-] $C_{IN,i}$, το κόστος απόσβεσης και αντικατάστασης μίας μονάδας σε € 
  \item[-] $C_{O\&M,i}$, το κόστος λειτουργίας και συντήρησης κάθε μονάδας σε €/εβδομάδα
  \item[-] $L_i$, ο χρόνος ζωής μίας μονάδας
\end{itemize}

Λαμβάνοντας υπόψην τα κόστη των παραπάνω μονάδων, το τελικό συνολικό κόστος διαμορφώνεται ως εξής:
\begin{equation}
    C=C_{B,CH}+C_{B,DIS}+ C_{H_2,CH}+C_{FC} \label{total_cost} 
\end{equation}

\section{Πρόβλημα βελτιστοποίησης}
Στόχος του μοντέλου είναι να υπολογίζεται η βέλτιστη κατάσταση του μικροδικτύου σε κάθε χρονικό βήμα, έχοντας ως στόχο την ελαχιστοποίηση του συνολικού κόστους λειτουργίας.

Για τον λόγο αυτό η αντικειμενική συνάρτηση είναι σχεδιασμένη έτσι ώστε να ελαχιστοποιείται το άθροισμα του συνολικού κόστους λειτουργίας του μικροδικτύου (\ref{total_cost}) και το κόστος της περίσσειας ενέργειας, αλλά και της χαμένης ενέργειας \cite{CAU2014820}:
\begin{equation}
    \begin{split}
        f=min \sum_{t=1}^{24} [C_{B,CH}(t) + C_{B,DIS}(t) + C_{H_2,CH}(t) + C_{FC}(t) + C_{EX}(t) + C_{UN}(t)]
    \end{split}
\end{equation}
όπου:
\begin{itemize}
%    \item[-] $ρ(s)$, η πιθανότητα να συμβεί το σενάριο s 
    \item[-] $C_{B,CH}(t)$, το κόστος φόρτισης της μπαταρίας
    \item[-] $C_{B,DIS}(t)$, το κόστος εκφόρτισης της μπαταρίας
    \item[-] $C_{Η_2,CH}(t)$, το κόστος φόρτισης της δεξαμενής υδρογόνου
    \item[-] $C_{FC}(t)$, το κόστος λειτουργίας της μονάδας κυψέλης καυσίμου υδρογόνου
    \item[-] $C_{EX}(t)$, το κόστος της περίσσειας ισχύος, το οποίο ισούται με 0.005 €/Wh
    \item[-] $C_{UN}(t)$, το κόστος της χαμένης ενέργειας, το οποίο ισούται με 0.005 €/Wh
\end{itemize}

Περιορισμοί προβλήματος ελαχιστοποίησης \cite{CAU2014820}:
\begin{itemize}
\item[1.] Ισοζύγιο ενέργειας:
Η ενέργεια που παράγεται στο μικροδίκτυο ισούται με την ενέργεια που καταναλώνεται. Ενέργεια μπορεί να δοθεί στο μικροδίκτυο από το φωτοβολταϊκό σύστημα, την ανεμογεννήτρια, τις μπαταρίες και τη μονάδα κυψελών καυσίμου υδρογόνου. Καταναλωτές είναι τα ευέλικτα φορτία, οι μπαταρίες κατά την περίοδο φόρτισης και η μονάδα ηλεκτρόλυσης. 

\begin{equation}
    \begin{split}
    P_{PV}(t) + P_{WT}(t) + P_{B,DIS}(t) \cdot Y_{B,DIS} + P_{FC}(t)
\cdot Y_{FC} + P_{UN}(t) \\
    = P_{LD}(t) + P_{B,CH}(t) \cdot Y_{B,CH} + P_{EL}(t) \cdot Y_{EL} + P_{EX}(t)  
    \end{split}
\end{equation}
        
όπου:
\begin{itemize}
    \item[-] $Y_i$ δυαδικές μεταβλητές που επιδεικνύουν την κατάσταση λειτουργίας μίας συσκευής (1=on, 0=off). Η συνολική ενέργεια που παράγεται ή/και εισάγεται στο μικροδίκτυο θα πρέπει να ισούται με την συνολική ενέργεια που καταναλώνεται ή/και εξάγεται, ανεξάρτητα του χρονικού βήματος και της προσέγγισης επίλυσης του προβλήματος. 
    \item[-] $P_{UN}$, η έλλειψη ενέργειας, undelivered power, σε $kW$
    \item[-] $P_{EX}$, η περίσσεια ενέργειας, excess power, σε $kW$
\end{itemize}

\item[2.] Όρια παραγωγής μονάδων:
\begin{align}
    0 \leq P_{B,CH}(t) \leq P_{B,MAX} \\
    0 \leq P_{B,DIS}(t) \leq P_{B,MAX} \\
    P_{EL,MIN} \leq P_{EL}(t) \leq P_{EL,MAX} \\
    P_{FC,MIN} \leq P_{FC}(t) \leq P_{FC,MAX} \\
    P_{UN}(t) \cdot P_{EX}(t) = 0 
\end{align}

\begin{align}
    0 \leq P_{UN}(t) \leq Mz(t) \\
    0 \leq P_{EX}(t) \leq M(1-z(t))   
\end{align}

\begin{align}
    P_{B,CH}(t) - P_{B,MAX}Y_{B,CH}(t) \leq 0 \\
    P_{B,DIS}(t) - P_{B,MAX}(1-Y_{B,CH}(t))\leq 0 \\
    P_{EL,MIN}Y_{EL}(t) - P_{EL}(t) \leq 0 \\
    P_{EL}(t) - P_{EL,MAX}Y_{EL}(t) \leq 0\\
    P_{FC,MIN}(1 - Y_{EL}(t)) - P_{FC}(t)\leq  0\\
    P_{FC}(t) - P_{FC,MAX}(1-Y_{EL}(t))\leq 0 \\
    P_{UN}(t) - M z(t) \leq 0 \\
    P_{EX}(t) - M (1-z(t))  \leq 0  
\end{align}

Η αποδοτικότητα των μπαταριών πέφτει σημαντικά με την υπερφόρτιση ή την εξάντλησή τους έως ότου αδειάσουν. Για τον λόγο αυτό είναι σημαντικό να τίθενται όρια στη μέγιστη ισχύ φόρτισης και εκφόρτισης. Παρόμοια συμπεριφορά παρουσιάζουν και η μονάδας ηλεκτρόλυσης και η μονάδα κυψελών καυσίμου υδρογόνου. Αν και μπορούν οι μονάδες να λειτουργήσουν και σε χαμηλή ισχύ, η αποδοτικότητά τους μειώνεται σημαντικά. Έτσι τίθενται και κάτω όρια για την λειτουργία των μονάδων.

\item[3.] Όρια αποθήκευσης: 
\begin{align}
    SOC_{MIN} \leq SOC(t) \leq SOC_{MAX} \\
    p_{H_2,MIN} \leq p_{H_2}(t) \leq p_{H_2,MAX} 
\end{align}

Οι μπαταρίες έχουν συνήθως μία ελάχιστη κατάσταση φόρτισης, η οποία προτείνεται από τον κατασκευαστή της εταιρείας, κάτω από την οποία δεν θα έπρεπε να λειτουργούν. Το χαρακτηριστικό αυτό επηρεάζει σημαντικά τον χρόνο ζωής των μπαταριών, καθώς μία υπερβολική αποφόρτιση προκαλεί έντονη υποβάθμιση  στον αριθμό των κύκλων φόρτισης μέχρι την αντικατάστασή τους.

Εφόσον δεν συμπεριλαμβάνεται συμπιεστής στη μονάδα αποθήκευσης υδρογόνου, η μέγιστη πίεση των δεξαμενών υδρογόνου είναι κι αυτή που λαμβάνεται από την μονάδα ηλεκτρόλυσης. Ο κατασκευαστής της μονάδας κυψελών καυσίμου υδρογόνου θέτει τον περιορισμό για την πίεση του παρεχόμενου υδρογόνου. Οπότε το όριο αυτό υποδεικνύει την ελάχιστη πίεση στη δεξαμενή αποθήκευσης υδρογόνου. 
\end{itemize} 



